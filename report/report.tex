\documentclass[twocolumn]{article}
\usepackage{hyperref}
\usepackage{url}
\usepackage{fancyhdr}

\pagestyle{fancy}

\title{Data Wrangling Coursework 2}
\author{Sam Dixon
    \\\href{mailto:40056761@live.napier.ac.uk}{40056761@live.napier.ac.uk}
    \\Edinburgh Napier University
    \\School Of Computing
    \\}

\date{April 2018}
\lhead{SET11521 - Data Wrangling}
\rhead{40056761}


\begin{document}


\maketitle


\section{Introduction}
This report describes the process of creating and evaluating a movie review
sentiment classification system.


\section{Data Processing}
The following preprocessing techniques were applied to the provided data file:
\begin{itemize}
    \item Removed capitalisation.
    \item Removed repeated lines.
    \item Shuffled the order of lines.
    \item Split sentiment and review into separate lists.
\end{itemize}


\section{Feature Extraction}
Once the data was processed, feature extraction was performed.
Text frequency-inverse document frequency (tf-idf) was utilised to calculate
weights for each unigram in the corpus \cite{tfidf}.
No minimum or maximum document frequency (cut-off) was applied to remove
common or uncommon corpus words - these methods could be explored further to
alter the accuracy of the classification.

After features were extracted, datasets could be constructed.
Both standard split, and k-folded datasets were generated to allow a
comparison of techniques.

The split dataset consisted of 9:1 training/testing split - the order of the
features was shuffled to prevent either the training or testing sets containing
only one class of review (positive or negative) \cite{split}.

The k-folded dataset was folded 10 times, with 9 folds being used to train a
classifier, and the remaining 1 fold used to test \cite{fold}.
Like the split dataset, the order of the features was shuffled to ensure
uniform distribution of sentiment classes.


\section{Classification Model}
Three different SVM kernels were tested in this project:

\begin{itemize}
    \item Linear.
    \item Radial Bias Function (RBF).
    \item Polynomial.
\end{itemize}
The default {\tt scikit-learn} implementation for each kernel was utilised,
further experimentation and turning of kernel parameters could be investigated
to improve their accuracy \cite{svm,kernel}.


\section{Evaluation}
Each SVM kernel's results was evaluated using the F1-score function
\cite{f1score}.  
F1-score is a commonly used metric for the comparison of
algorithm accuracies.
The F1-score of a system is calculated based off of its Precision and
Recall.
F1-score is typically a more accurate measure of an algorithm's performance
then simply comparison the number correct and incorrect predictions.

Each of the three kernels was trained and tested with the split and k-folded
datasets 10 times, and the average F1-score for each was calculated. 
Each dataset was randomly generated from the tf-idf feature list before
training and testing.
The results for each kernel and dataset is displayed in Table \ref{table_res}.

\begin{table}[]
    \centering
        \begin{tabular}{|l|l|l|}
        \hline
        \textbf{Kernel} & \textbf{Dataset} & \textbf{Average F1-Score} \\ \hline
        Linear          & Split            & 0.963636363636            \\ \hline
        Linear          & Fold             & 0.961481481481            \\ \hline
        RBF             & Split            & 0.536026936027            \\ \hline
        RBF             & Fold             & 0.540740740741            \\ \hline
        Polynomial      & Split            & 0.52861952862             \\ \hline
        Polynomial      & Fold             & 0.540740740741            \\ \hline
        \end{tabular}
    \caption{Average kernel F1-score over 10 tests.}
    \label{table_res}
\end{table}


\section{Results}
As can be seen, the k-folded dataset is only marginally superior to the split
alternative.
The similarity in results is likely due to the limited size of the data set.

In terms of kernel performance, the linear SVM classifier obviously
outperforms the alternative methods.
As stated, this may be due to the lack tuning of the RBF and Polynomial
kernels, which feature a wide array of variables that can be altered to
improve accuracy.


\section{Conclusion}
From the results of testing each kernel and dataset the following observations
can be made:

With a limited amount of data, the manner used to divide the dataset is of
little consequence, provided numerous test are conducted.

In terms of SVM kernel choice, linear produces satisfactory results with little
to no tuning requirements.
The RBF and polynomial kernels may be able to produce similar or superior
results to the linear kernel, provided adequate time is available to tune
and test the kernel's parameters.

\noindent
Project available at: \url{https://github.com/neaop/set11521_coursework_2}.

\bibliography{report}
\bibliographystyle{ieeetr}

\end{document}
